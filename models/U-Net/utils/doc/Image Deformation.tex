\documentclass[12pt]{article}

\usepackage{amsmath, amsthm}
\usepackage{amsfonts}
\usepackage{amssymb}
\usepackage{mathrsfs}
\usepackage{xeCJK}
\usepackage{indentfirst}
\usepackage{fontspec}
\usepackage{setspace}
\usepackage{geometry}
\usepackage{graphicx}
\usepackage{algorithm}
\usepackage{algorithmicx}
\usepackage{algpseudocode}
\usepackage{color}
\usepackage{framed}
\usepackage[colorlinks, linkcolor=black]{hyperref}

\renewcommand{\baselinestretch}{1.5}
\geometry{left=2.5cm,right=2.5cm,top=2.5cm,bottom=2.5cm}
\setlength\parindent{2em} 
\renewcommand{\today}{\number\year 年 \number\month 月 \number\day 日}
\newtheorem{lemma}{\hspace{2em}引理}
\newtheorem{theorem}{\hspace{2em}定理}
\newtheorem{defination}{\hspace{2em}定义}

\makeatletter
\renewenvironment{proof}[1][\hspace{2em}证]{
	\par%
	\def\FrameCommand{\fboxsep=\FrameSep \colorbox{shadecolor}}%
	\MakeFramed {\FrameRestore}%
	\pushQED{\qed}%
	\linespread{1.3}\selectfont%
	\CJKfamily{kai} \topsep6\p@\@plus6\p@\relax%
	\trivlist%
	\item\relax%
	{\CJKfamily{hei}#1\hspace{1em}}\hspace\labelsep\ignorespaces
}{%
	\popQED\endtrivlist\@endpefalse\endMakeFramed%
}
\makeatother

\definecolor{shadecolor}{rgb}{0.97,0.97,0.97}

\begin{document}
\author{张建伟}
\date{\today}
\title{Image Deformation Using Moving Least Squares\\ 阅读笔记}
\maketitle

\section{Moving Least Squares Deformation}
\begin{itemize}
	\item $p$: 一列控制顶点.
	\item $q$: 控制顶点变换后的坐标. 
\end{itemize}
给定图上的一点$v$, 求解一个最优的仿射变换来最小化
\begin{equation}
	\sum_i w_i|l_v(p_i)-q_i|^2,
\end{equation}
其中$p_i$和$q_i$都是行向量, 每行的分量为点的坐标, 权重$w_i$有如下的形式
$$
w_i = \frac{1}{|p_i-v|^{2\alpha}}.
$$
因为该最小二乘问题中的权重$w_i$独立于$v$变形后的点, 所以我们称之为{\it 移动最小二乘最小化}. 对于不同的$v$, 可以得到不同的变换$l_v(x)$. 由于$l_v(x)$是仿射变换, 所以可以写成
\begin{equation}
	l_v(x) = xM+T.
\end{equation}
令原始的优化函数对$T$求偏导数并令其为$0$, 解出
\begin{equation*}
	T = q^* - p^*M,
\end{equation*}
其中$p^*$和$q^*$是原来一系列控制顶点的加权质心, 
\begin{equation*}
	p^* = \frac{\sum_i w_ip_i}{\sum_i w_i},\qquad 	q^* = \frac{\sum_i w_iq_i}{\sum_i w_i}.
\end{equation*}
所以有
\begin{equation}
	l_v(x) = (x-p^*)M+q^*.
\end{equation}
所以原优化函数可以修改为
\begin{equation}\label{eq01}
	\sum_i w_i|\hat{p}_iM-\hat{q}_i|^2,
\end{equation}
其中$\hat{p}_i=p_i-p^*,\;\hat{q}_i=q_i-q^*$, 考虑二维图像时, $M$就是一个$2\times2$的矩阵. 

\subsection{Affine Deformation}
要找一个仿射变换来极小化方程(\ref{eq01}), 直接用古典方法求解优化问题得
\begin{equation*}
	M = \left(\sum_i\hat{p}_i^{\top}w_i\hat{p}_i\right)^{-1}\;\sum_j\hat{p}_j^{\top}w_j\hat{q}_j.
\end{equation*}
从而我们可以写出仿射变换的表达式
\begin{equation}
	f_a(v) = (v-p^*)\left(\sum_i\hat{p}_i^{\top}w_i\hat{p}_i\right)^{-1}\;\sum_j\hat{p}_j^{\top}w_j\hat{q}_j+q^*.
\end{equation}
又因为$p_i$是固定的, 所以上式可以变为
\begin{equation*}
	f_a(v) = \sum_jA_j\hat{q}_j + q^*,
\end{equation*}
其中$A_j$可以预计算
\begin{equation*}
	A_j = (v-p^*)\left(\sum_i\hat{p}_i^{\top}w_i\hat{p}_i\right)^{-1}w_j\hat{p}_j^{\top}.
\end{equation*}

直接做仿射变换会存在一些问题:原图中的网格点阵是整齐排列的, 变换到目标图像中后便不再整齐排列,由于是浮点数运算, 所以有一些点会变换到目标图的同一个点上, 而目标图的有一些点没有任何点从原图变换过来,这就会导致变换之后的图像产生白色的镂空。解决该问题的一个比较简单的办法就是对原始图像做逆变换,这相当于在目标图像的网格点阵上计算原图中对应的点,即已知$f_a(v)$求解对应的$v$。计算公式如下:
\begin{equation}
	v = (f_a(v) - q^*)\left(\sum_j\hat{p}_j^{\top}w_j\hat{q}_j\right)^{-1}\left(\sum_i\hat{p}_i^{\top}w_i\hat{p}_i\right) + p^*
\end{equation}

\subsection{Similarity Deformation}
实际上仿射变换包含了非一致性的平移和放缩,实际中的许多物体并不会产生这么复杂的变化。相似变换是仿射变换的一个子类,仅包含平移、旋转和一致的放缩。为了满足相似变换的性质,我们限制矩阵$M$满足$M^{\top}M=\lambda^2I, \exists\lambda$。如果$M$是分块矩阵,有$M=(M_1,\;\;M_2)$的形式,其中$M_1, M_2$都是长度为$2$的列向量,那么对于$M$的限制可以变为$M_1^{\top}M_1=M_2^{\top}M_2=\lambda^2$,并且$M_1^{\top}M_2=0$。这个限制意味着$M_2=M_2^{\bot}$,其中$\bot$是一个作用于二维向量的算子使得$(x, y)^{\bot}=(-y, x)$。这样原来的目标方程(\ref{eq01})可以修改为
\begin{equation}
	\sum_i w_i\left|\left(\begin{matrix}
	\hat{p}_i \\ -\hat{p}_i^{\bot}
	\end{matrix}\right)M_1-\hat{q}_i^{\top}\right|^2.
\end{equation}
该二次方程有唯一的最优值,从而可以得到最优值点$M$
\begin{equation}\label{eq02}
	M = \frac{1}{\mu_s}\sum_i w_i\left(\begin{matrix}
	\hat{p}_i \\ \hat{p}_i^{\bot}
	\end{matrix}\right)\left(\begin{matrix}
	\hat{q}_i^{\top} & \hat{q}_i^{\bot\top}
	\end{matrix}\right),
\end{equation}
其中$\mu_s=\sum_i w_i\hat{p}_i\hat{p}_i^{\top}$。从而得到最终的变换公式
\begin{equation*}
	f_s(v) = \sum_i\hat{q}_i\left(\frac{1}{\mu_s}A_i\right)+q^*,
\end{equation*}
其中$A_i$是
\begin{equation}\label{eq03}
	A_i = w_i\left(\begin{matrix}
	\hat{p}_i \\ \hat{p}_i^{\bot}
	\end{matrix}\right)\left(\begin{matrix}
	v-p^* \\ -(v-p^*)^{\bot}
	\end{matrix}\right)^{\top}.
\end{equation}
类似的我们可以得到逆相似变换的公式
\begin{equation}
	v = \mu_s(f_s(v) - q^*)\left(\begin{matrix}
	\Delta \\ \Delta^{\bot}
	\end{matrix}\right)^{\top} + p^*,
\end{equation}
其中
$$
\Delta=\sum_i\hat{q}_iw_i\left(\begin{matrix}\hat{p}_i\\\hat{p}_i^{\bot}\end{matrix}\right).
$$

\subsection{Rigid Deformation}
进一步地,我们要求变换中不包括一致放缩,即限制变为$M^{\top}M=I$。先给出一个定理,这个定理说明了刚性变换和相似变换的关系。
\begin{theorem}
	令$C$是可以极小化如下相似问题的矩阵
	\begin{equation*}
		\min_{M^{\top}M=\lambda^2I}\sum_iw_i\left|\hat{p}_iM-\hat{q}_i\right|.
	\end{equation*}
	如果$C$写成$\lambda R$的形式,$R$是一个旋转矩阵,$\lambda$是一个标量,那么旋转矩阵$R$极小化如下的刚性问题
	\begin{equation*}
		\min_{M^{\top}M=I}\sum_iw_i\left|\hat{p}_iM-\hat{q}_i\right|.
	\end{equation*}
\end{theorem}
定理证明略去,可以参考原文中的$Appendix A$。

根据定理我们知道刚性变化恰好就是方程(\ref{eq02}),除了把其中的$\mu_s$替换为$\mu_r$
\begin{equation*}
	\mu_r = \sqrt{\left(\sum_iw_i\hat{q}_i\hat{p}_i^{\top}\right)^2 + \left(\sum_iw_i\hat{q}_i\hat{p}_i^{\bot\top}\right)^2}.
\end{equation*}
令
\begin{equation*}
	\overrightarrow{f}_r(v)=\sum_i\hat{q}_iA_i,
\end{equation*}
其中$A_i$由式(\ref{eq03})定义,最后的变换公式为
\begin{equation}
	f_r(v)  = \left|v-p^*\right|\frac{\overrightarrow{f}_r(v)}{\left|\overrightarrow{f}_r(v)\right|} + q^*.
\end{equation}
上述变换公式不易求得其逆变换,所以近似地使用如下逆变换
\begin{equation*}
	v  = \left|f(v)-q^*\right|\frac{\overrightarrow{g}_r(v)}{\left|\overrightarrow{g}_r(v)\right|} + p^*.
\end{equation*}
其中
\begin{equation*}
	\overrightarrow{g}_r(v) = (f(v)-q^*)\left(\begin{matrix}	\Delta \\ \Delta^{\bot}	\end{matrix}\right)^{-\top}
\end{equation*}







\end{document}