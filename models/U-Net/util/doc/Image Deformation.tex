\documentclass[12pt]{article}

\usepackage{amsmath, amsthm}
\usepackage{amsfonts}
\usepackage{amssymb}
\usepackage{mathrsfs}
\usepackage{xeCJK}
\usepackage{indentfirst}
\usepackage{fontspec}
\usepackage{setspace}
\usepackage{geometry}
\usepackage{graphicx}
\usepackage{algorithm}
\usepackage{algorithmicx}
\usepackage{algpseudocode}
\usepackage{color}
\usepackage{framed}
\usepackage[colorlinks, linkcolor=black]{hyperref}
\setCJKfamilyfont{hei}{SimHei}
\setCJKfamilyfont{kai}{KaiTi}
\setCJKmainfont[BoldFont={SimHei}, ItalicFont={KaiTi}]{SimSun}

\renewcommand{\baselinestretch}{1.5}
\geometry{left=2.5cm,right=2.5cm,top=2.5cm,bottom=2.5cm}
\setlength\parindent{2em} 
\renewcommand{\today}{\number\year 年 \number\month 月 \number\day 日}
\newtheorem{lemma}{\hspace{2em}引理}
\newtheorem{theorem}{\hspace{2em}定理}
\newtheorem{defination}{\hspace{2em}定义}

\makeatletter
\renewenvironment{proof}[1][\hspace{2em}证]{
	\par%
	\def\FrameCommand{\fboxsep=\FrameSep \colorbox{shadecolor}}%
	\MakeFramed {\FrameRestore}%
	\pushQED{\qed}%
	\linespread{1.3}\selectfont%
	\CJKfamily{kai} \topsep6\p@\@plus6\p@\relax%
	\trivlist%
	\item\relax%
	{\CJKfamily{hei}#1\hspace{1em}}\hspace\labelsep\ignorespaces
}{%
	\popQED\endtrivlist\@endpefalse\endMakeFramed%
}
\makeatother

\definecolor{shadecolor}{rgb}{0.97,0.97,0.97}

\begin{document}
\author{张建伟}
\date{\today}
\title{Image Deformation Using Moving Least Squares\\ 阅读笔记}
\maketitle

\section{Moving Least Squares Deformation}
\begin{itemize}
	\item $p$: 一列控制顶点.
	\item $q$: 控制顶点变换后的坐标. 
\end{itemize}
给定图上的一点$v$, 求解一个最优的仿射变换来最小化
\begin{equation}
	\sum_i w_i|l_v(p_i)-q_i|^2,
\end{equation}
其中$p_i$和$q_i$都是行向量, 每行的分量为点的坐标, 权重$w_i$有如下的形式
$$
w_i = \frac{1}{|p_i-v|^{2\alpha}}.
$$
因为该最小二乘问题中的权重$w_i$独立于$v$变形后的点, 所以我们称之为{\it 移动最小二乘最小化}. 对于不同的$v$, 可以得到不同的变换$l_v(x)$. 由于$l_v(x)$是仿射变换, 所以可以写成
\begin{equation}
	l_v(x) = xM+T.
\end{equation}
令原始的优化函数对$T$求偏导数并令其为$0$, 解出
\begin{equation*}
	T = q^* - p^*M,
\end{equation*}
其中$p^*$和$q^*$是原来一系列控制顶点的加权质心, 
\begin{equation*}
	p^* = \frac{\sum_i w_ip_i}{\sum_i w_i},\qquad 	q^* = \frac{\sum_i w_iq_i}{\sum_i w_i}.
\end{equation*}
所以有
\begin{equation}
	l_v(x) = (x-p^*)M+q^*.
\end{equation}
所以原优化函数可以修改为
\begin{equation}\label{eq01}
	\sum_i w_i|\hat{p}_iM-\hat{q}_i|^2,
\end{equation}
其中$\hat{p}_i=p_i-p^*,\;\hat{q}_i=q_i-q^*$, 考虑二维图像时, $M$就是一个$2\times2$的矩阵. 

\subsection{Affine Deformation}
要找一个仿射变换来极小化方程(\ref{eq01}), 直接用古典方法求解优化问题得
\begin{equation*}
	M = \left(\sum_i\hat{p}_i^{\top}w_i\hat{p}_i\right)^{-1}\;\sum_j\hat{p}_j^{\top}w_j\hat{q}_j.
\end{equation*}
从而我们可以写出仿射变换的表达式
\begin{equation}
	f_a(v) = (v-p^*)\left(\sum_i\hat{p}_i^{\top}w_i\hat{p}_i\right)^{-1}\;\sum_j\hat{p}_j^{\top}w_j\hat{q}_j+q^*.
\end{equation}
又因为$p_i$是固定的, 所以上式可以变为
\begin{equation*}
	f_a(v) = \sum_jA_j\hat{q}_j + q^*,
\end{equation*}
其中$A_j$可以预计算
\begin{equation*}
	A_j = (v-p^*)\left(\sum_i\hat{p}_i^{\top}w_i\hat{p}_i\right)^{-1}w_j\hat{p}_j^{\top}.
\end{equation*}


















\end{document}